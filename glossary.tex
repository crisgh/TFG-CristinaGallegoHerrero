\newacronym{tfg}{TFG}{Trabajo de Fin de Grado}

\newacronym{pdf}{PDF}{Portable Document Format}

\newacronym{gui}{GUI}{Graphic User Interface (Interfaz Gráfica de Usuario)}

\newacronym{eer}{EER}{Extended Entity-Relationship}

\newacronym{ide}{IDE}{Integrated Development Enviroment (Entorno de Desarrollo Integrado)}

\newacronym{oss}{OSS}{Open Source Software (software de código abierto)}

\newacronym{bbdd}{BBDD}{Base de Datos}

\newacronym{rdd}{RDD}{Resilient Distributed Datasets (Conjunto de Datos Distribuido y Flexible)}

\newacronym{xml}{XML}{Extensible Markup Language (Lenguaje de Marcado Extensible)}

\newacronym{csv}{CSV}{Comma Separated Values (Valores Separados por Comas)}

\newglossaryentry{sql}{
  name={SQL},
  description={Structured Query Language, traducido en español como lenguaje de consulta estructurada, es un lenguaje de dominio específico utilizado en programación, diseñado para administrar y recuperar información de sistemas de gestión de bases de datos relacionales.}
}


\newglossaryentry{json}{
  name={JSON},
  description={JavaScript Object Notation, traducido como notación de objeto de JavaScript, es un formato basado en el uso de texto estándar para representar datos estructurados. Aunque se basa en sintaxis JavaScript puede ser utilizado independientemente y muchos frameworks de programación poseen la capacidad de leer y generar este tipo de objetos.}
}

\newglossaryentry{plugin}{
  name={Plug-in},
  text={plug-in}, % Para que en el texto salga en minúscula
  description={Plugin o complemento, es una aplicación o programa que permite agregar nuevas funcionalidades a la aplicación principal sin tener que modificar el programa original.}
}
